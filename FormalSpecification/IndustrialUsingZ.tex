\documentclass{article}
\usepackage{CJKutf8}
\usepackage{setspace}
\usepackage{geometry}
\usepackage{fancyhdr}
\usepackage{lastpage}
\usepackage{layout}  
\pagestyle{fancy}
\rhead{page \thepage\ of \pageref{LastPage}}
\chead{}
\lhead{\bfseries Formal Specification and Verification Homework}

\renewcommand{\headrulewidth}{0.4pt}
\renewcommand{\footrulewidth}{0.4pt}
\newgeometry{left=3.5cm,right=3.5cm,bottom=2cm,top=2cm}
\author{16121731 皮怀雨}
\title{软件工程形式化方法论文总结(三)}
\begin{document}
\begin{CJK}{UTF8}{gbsn}
\maketitle
\begin{spacing}{1.4}
\renewcommand{\CJKglue}{\hskip 0.5pt}

通过阅读Industrial Use of Formal Methods,了解了形式化方法在工业中的应用。形式化方法在实际工程项目中,存在一些问题和误解。文章也对形式化方法的未来的发展进行了分析,它在工业中应用会越来越广泛。

在大型的工业软件项目中,形式化方法的运用还是非常的有限,但是在一些项目中也起到了很好的作用。形式化方法在软件工程项目中使用不够广泛有很多原因。在软件技术不断发展的今天,有些软件系统已经变得非常复杂,现有的形式化工具大多是在学术研究的过程中创造的,但是这些工具离工业生产还有很大的距离。还有就是在教育和标准方面,各国对软件形式化的重视程度不同,而且也没有一个统一的标准。正是这种状况的存在,使得我们难以对复杂的软件系统做全面的分析和证明。形式化方法对大型软件工程来说依然非常具有价值,因为一个软件系统可能会大到难以去证明它的正确性,那么从一开始就正确的设计整个系统对后来的系统开发至关重要。而形式化方法的引入就将大大提高软件设计的简洁性和可读性,可以极大地减少错误。事实上,大型的软件系统也有使用形式化方法,但是它的作范围非常有限,一般只是对系统做一个形式化的表示,使得设计更加清晰和简洁。而形式化方法的证明并不是所有工程都必须用到的,证明正确性是耗费很大的环节,所以从成本和时间的角度考虑,我们可以部分使用或者不使用证明。对形式化方法的巧妙使用,而不是生搬硬套,能够降低软件中错误提高软件开发的效率,减少不必要的、高成本的修改。

我们应该了解这种更加严谨的用数学语言表示的方法,因为它有对工业化的软件有着很多的益处。Praxis plc的总裁说:“如果你不能用数学的形式来表达你所设计的系统的行为,那么你就不能真正的理解整个系统。”这说明使用形式化方法来表达软件系统的行为,可以使得我们对开发的软件有更好的理解。就像之前说的一样,巧妙地在软件设计过程中使用形式化方法,是非常有效的。在软件设计和开发中,软件的接口变得非常重要,它是整个软件的基础方法。那么,如果我们在接口的设计中引入形式化方法,将对我们后续的软件开发带来极大的益处。在形式化方法使用的过程中,我们应该注意一些问题,从而高效的使用它。首先,我们应该选取一个合适的形式化语言,比如B-Method相对于Z来说工具更加丰富。还有我们应该注意适当的使用形式化方法,而不是强行在不合适的地方使用,因为在软件工程中我们应该控制整个软件的成本,选取合适的方法,保证项目的质量和进度。最后,形式化方法并不能代替软件测试,因为我们不能保证我们的形式化表示没有错误,特别是在没有证明的情况下。在软件开发完成后我们依然要进行软件测试以保证软件的稳定和安全。


\end{spacing}
\end{CJK}
\end{document}