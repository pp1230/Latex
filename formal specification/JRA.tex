\documentclass{article}
\usepackage{CJKutf8}
\usepackage{setspace}
\usepackage{geometry}
\newgeometry{left=3cm,bottom=1cm}
\author{16121731 皮怀雨}
\title{工业开发中的形式化方法}
\begin{document}
\begin{CJK}{UTF8}{gbsn}
\maketitle
\par
\begin{spacing}{1.4}
\renewcommand{\CJKglue}{\hskip 0.5pt}
通过对Jean-Raymond Abrial的报告“工业开发中形式化方法”的阅读,了解了一种新的软件工程开发方式,从中也理解了一种更加严谨的工程方法。它对于传统的开发方式有了一个全新的颠覆,有着更加严谨的规范。
在软件开发日益复杂的今天,软件的稳定性、安全性和复用性越来越重要。但是在常规的软件开发中,往往对于需求分析和建模投入的资源有限,主要通过开发过程的控制和后续测试的反复,来提高软件的稳定性和安全性,但是这种方式往往存在着很多的安全隐患。在软件开发的过程中和测试后对系统进行修改和更新付出了大量的时间精力,代价非常大。对于这种模式,我们应该进行一些反思,是否能使用更加严谨的和可控的工程方法控制软件的品质?在成熟的工程领域,往往可以通过数学的计算和模型来验证工程是否达到各项指标。
\par
在很多对安全性要求较高的领域,比如轨道交通和航天,采用更加可靠的开发方法非常必要。如果采用形式化方法对软件进行需求分析、抽象模型然后转化为具体模型和自动生成代码的方法开发,就会得到相对正确的代码(对于需求和模型)。形式化开发的过程主要分为3个步骤,首先是进行需求分析,这是所有软件工程的必要阶段,而且非常重要。准确的需求分析决定了软件最终是否符合用户的要求,在有了准确的需求之后,形式化方法和常规的开发有所方法不同。然后就是进行逐步求精建立抽象模型,在这个过程中,需要大量的人工参与,构建好抽象模型是形式化方法中耗费最大的步骤。对比常规的开发方法,虽然也需要构建模型,画出用例图、流程图和类图等为后来开发做参考,但是在这个过程中耗费的人力在整个软件工程中是较少的。从得到的抽象模型,我们可以得到具体模型,这个过程不再不要需求文档,而且这个过程的人工参与不如之前的广泛。最后一步是模型自动翻译成为可执行代码,这一阶段不需要人工的参与。而在常规的软件工程过程中,编码往往是最为耗时的,而且会出现很多错误。但是,自动翻译也是现在形式化方法很大的挑战,如何自动产生高质量的可执行代码还是一个难题。
\par
按照软件工程形式化方法,软件工程师通过需求分析和逐步求精建立模型,可以最终翻译成为一个安全和可复用的系统。如果软件工程师只要完成相关的模型构建,那么就不需要他们去具体的编写代码,这样降低了软件开发过程中对不同新技术的掌握成本。软件工程师只需要掌握形式化方法,就可以构建正确的软件。使用形式化方法构建软件系统,省去了软件开发繁重的编码过程,而且降低了系统的错误。
\end{spacing}
\end{CJK}
\end{document}