\documentclass{article}
\usepackage{CJKutf8}
\usepackage{setspace}
\usepackage{geometry}
\usepackage{fancyhdr}
\usepackage{lastpage}
\usepackage{layout}  
\pagestyle{fancy}
\rhead{page \thepage\ of \pageref{LastPage}}
\chead{}
\lhead{\bfseries Formal Specification and Verification Homework}

\renewcommand{\headrulewidth}{0.4pt}
\renewcommand{\footrulewidth}{0.4pt}
\newgeometry{left=3.5cm,right=3.5cm,bottom=2cm,top=2cm}
\author{16121731 皮怀雨}
\title{软件工程形式化方法论文总结(二)}
\begin{document}
\begin{CJK}{UTF8}{gbsn}
\maketitle
\begin{spacing}{1.4}
\renewcommand{\CJKglue}{\hskip 0.5pt}

通过阅读Formal Specification and Documentation Using Z Chapter 1,对形式化的表示方法和Z语言的设计有了一定的理解。通过实际的例子,也了解到Z语言在实际运用中应该注意的问题。

形式化表示方法相比自然语言具有显著的优势,但也有一些缺点。自然语言的表述常常含糊不清、具有二义性,让人产生误解。虽然在很多场景下,自然语言的性质有它的优势,但是在一份软件开发文档中,这显然会导致众多的错误,带来很多额外的消耗。形式化方法采用数学符号表示,不但简洁,而且表述准确。当然,它也存在问题。人们更加愿意使用易于理解的自然语言来编写文档,因为形式化语言有着大量的难懂的数学符号。对于一个未经严格数学训练的软件工程师或者客户来说,这会让人难以理解。所以我们需要一种既严格又易于理解的语言来编写让人满意的软件系统文档。

Z就是既简洁又易于理解的形式化语言。如果我们只是用数学符号来描述一个系统,那么如此大量的符号不但让人难以理解,达不到描述系统的目的,还会让整个设计文档难以管理。所以Z不但包含了大量的数学符号,他还提供了schema来帮助描述系统的结构。这样的设计使得文档不全是形式化符号的填充,而包含了更多易于理解的schema。这也是Z为什么能用来设计系统的最为核心的内容。Z设计的初衷就是面向用户的,而不是面向计算机的可执行语言。相比C/C++等可执行的计算机语言,Z注重描述的是一个系统是什么,以及如何运行,而可执行语言需要解决的是,如何才能设计出计算机执行的代码。所以从这个角度来说,Z不像函数式语言,也不是为了得到结果而设计的。我们使用Z是为了简洁而准确地描述一个复杂系统,让使用系统的人都能很好的理解它的设计。

使用Z语言的过程中应该注意一些问题。就像其他的众多编程语言一样,Z语言的编写也可以有很多种风格。在实践中,我们发现相比函数式编程风格来说,基于状态或模型的编程风格更有效率。Z语言使用也有一些一般的步骤,首先应该将系统表示为一些集合,然后定义一些抽象的状态、关系和方法。之后赋予各种状态一些初始值,这些状态会随着系统的运行而改变。系统的运行进程在Z中是原子性的,不能在在一个进程运行过程中再运行另外一个进程。最后进行各种状态和过程的逐步求精。

在实际的运用中,Z语言展现了它的优势和劣势。在Oxford University 的DCS项目中,使用Z语言进行了网络服务的设计。设计了包括用户手册和实现手册在的内面向不同对象的设计手册。用户手册包含了简洁、服务的抽象状态和服务进程的设计,其中进程运行的的设计包含了抽象描述、定义和询问阶段。实现手册主要是面向开发人员,主要是帮助开发人员理解整个系统,而不是去证明它的正确性。用户手册和开发手册中包含了相同的部分,就是共有的服务框架。通过Z语言的设计,找出了共有的部分,减少了重复设计,降低了文档的复杂度。在Unix软件的例子中,发现Z语言不但可以设计系统,还可以对现有的系统进行清晰地描述,来证明系统的正确性,修改系统中的错误。以Z语言为基础,我们还可以写出其他的语言如CSP来处理并发问题。说到并发问题,Z语言有它的不足。它本身不擅长处理并发问题,而且Z语言的相关工具集依然未被工业界广泛采用。在实际的使用中,我们往往使用Z来设计系统,而不是用它证明系统的正确性。

总的来说,使用Z语言我们可以更好的理解系统的运行过程,在系统实现之前发现和解决问题。Z语言还可以通过数学符号来表示系统的运行状态,对实现做更进一步的精化。

\end{spacing}
\end{CJK}
\end{document}